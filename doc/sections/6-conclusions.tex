\section{Conclusions}
Networks are obtained by first generating 10000 nodes with $k_i$ stubs ($0 \leq i < n$). Each $k_i$ is selected randomly using a Poisson or a power law distribution. The $l_i$ interaction events a node has are then distributed among its $k_i$ stubs. Again, each $l_i$ is randomly picked using a Poisson or power law distributions, or they are just $\delta \cdot k_i$. We refer to this as the delta distribution, though it is just a multiplication of $k_i$. Moreover, the interactions of a node are distributed among the stubs in the following way:
\begin{enumerate}
    \item Generate one real number $r_j$ between 0 and 1 for each stub ($0 \leq j < k_i$).
    \item Calculate the total sum. $$total = \sum_{j=0}^{k_i}r_j$$
    \item Each stub is then equal to $$\lfloor \frac{(l_i-k_i)\cdot r_j}{total}\rfloor$$
    With this, all (or almost all) the interactions are distributed.
    \item Distribute randomly the remaining interactions.
\end{enumerate}

Except the power law, all the distributions and random number generators used in this work are the ones implemented in the well-known \textit{random} standard library of C++. Random numbers follow a uniform distribution.