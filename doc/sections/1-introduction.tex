\section{Introduction}
Understanding how infectious diseases spread, has public health and ecological implications. The contact structure between hosts is known to have a key influence on disease spread. However, most studies assume that all types of contacts are identical, when in reality some individuals interact more strongly than others. This can be clearly seen in, for example, sexual-contact networks, where the number of sex acts is not equal for all partners. These differences between weights among partners, generates an heterogenous network and this heterogeneity can affect by speeding up or slowing down an epidemic spread depending on how strongly connected the hosts are.

In this project we are going to generate various networks that try to imitate the real world, changing the heterogeneity of them and seeing how two different infection model spread on those. This is a replica and extension of a paper \cite{10.1371/journal.pcbi.1003352}, there they tell how to generate the networks but they only test it with SIR model and we extend it by testing also SIS model.